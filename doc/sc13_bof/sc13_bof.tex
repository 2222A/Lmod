\documentclass{beamer}

\usetheme[headernav]{TACC} %%Drop the 'headernav' if you don't like
                           %%the stuff at the top of your slide

\usepackage{amsmath,amssymb,amsthm}
\usepackage{alltt}
\usepackage{graphicx}

\title{Lmod BoF at SC'13}


\author{Robert McLay}
\institute{The Texas Advanced Computing Center}

\date{November 20, 2012}  %% Use this if you want to fix the date in
                        %% stone rather than use \today

\newcommand{\bfnabla}{\mbox{\boldmath$\nabla$}}
\newcommand{\laplacian}[1]{\bfnabla^2 #1}
\newcommand{\grad}[1]{\bfnabla #1}
\newcommand{\tgrad}[1]{\bfnabla^T #1}
\newcommand{\dvg}[1]{\bfnabla \cdot #1}
\newcommand{\curl}[1]{\bfnabla \times #1}
\newcommand{\lap}[1]{\bfnabla^2 #1}

\begin{document}

\begin{frame}
  \titlepage
\end{frame}

\section{Introduction}

\begin{frame}{Introduction}
  \begin{itemize}
    \item What is Lmod?
    \item Recent Features
    \item Topics For Discussion
    \item For more information.
  \end{itemize}
\end{frame}


\begin{frame}{What is Lmod?}
  \begin{itemize}
    \item Lmod is yet another Environment Module System.
    \item Similar to Cmod, TCL/C Modules, Softenv, DOTKIT
    \item Lmod Features:
      \begin{itemize}
        \item Supports Multiple Compiler/MPI Stacks.
        \item Automatically swaps dependent modules: PETSc, fftw3.
        \item Reads TCL module files or Lua module files.
        \item Implemented in Lua.
      \end{itemize}
  \end{itemize}
\end{frame}

\section{Recent Developments}

\begin{frame}{New Features}
  \begin{itemize}
    \item Module properties.
    \item Modulefiles are now read through a ``sandbox''
    \item Full support of Category/name/version module name convention
      (with nesting as complex as you like).
    \item Sticky modules: unloaded or purged with ``--force'' flag.
    \item Improvements when searching for default modules.
    \item Allow duplicate paths.
    \item New function ``\texttt{pushenv}''
  \end{itemize}
\end{frame}
\begin{frame}{Previous Features}
  \begin{itemize}
    \item Case-insensitive searching with ``module spider''
    \item spider, keyword, avail and help now use \texttt{\$PAGER}
    \item A standard way to support site packages.
    \item \texttt{getdefault/setdefault} becoming \texttt{save/restore}
    \item Module files can have properties: MIC, Apps/Lib, ...
    \item module wrapper command for the typing challenged: ``ml''
    \item Version Parsing: 5.6 is now older than 5.10
    \item Mailing list: lmod-users@lists.sourceforge.net
  \end{itemize}
\end{frame}

\begin{frame}{Save / Restore}
  \begin{itemize}
    \item \texttt{getdefault/setdefault} becoming \texttt{save/restore}
    \item \texttt{module restore} load user's default or system
      default when user's default doesn't exist
    \item \texttt{reset} and \texttt{getdefault/setdefault} will be deprecated.
  \end{itemize}
\end{frame}

\begin{frame}[fragile]
    \frametitle{Module Properties}
  \begin{itemize}
    \item TACC is deploying Stampede with MIC accelerators
    \item Some modules will be ``MIC'' aware: mkl, fftw3, phdf5, ...
    \item Lmod will decorate these modules:
  {\tiny
    \begin{alltt}
  1) unix/unix     3) ddt/ddt       5) mpich2/1.5    7) {\color{blue}phdf5/1.8.9 (m)}
  2) intel/13.0    4) {\color{red}mkl/mkl (*)}   6) petsc/3.2     8) PrgEnv

  Where:
   {\color{blue}(m)}:  module is build natively for MIC
   {\color{red}(*)}:  module is build natively for MIC and offload to the MIC

   ------
   add_property("arch","mic")              -- > phdf5
   add_property("arch","mic:offload")      -- > mkl
    \end{alltt}
}
  \item What properties would you like to support?
  \end{itemize}
\end{frame}


\begin{frame}{For those who can't type: ``\texttt{ml}''}
  \begin{itemize}
    \item \texttt{ml} is a wrapper:
      \begin{itemize}
        \item With no argument: \texttt{ml} means \texttt{module list}
        \item With a module name: \texttt{ml foo} means \texttt{module
            load foo}.
        \item With a module command: \texttt{ml spider} means
          \texttt{module spider}.
      \end{itemize}
    \item See \texttt{ml --help} for more documentation.
    \item With \texttt{ml}, you never need to spell ``\texttt{moduel, moudule, module}''
      correctly again.
  \end{itemize}
\end{frame}

\begin{frame}{Module version sorting}
  \begin{itemize}
    \item Old way lexicographically sort: 5.6 is newer that 5.10
    \item New way 5.10 is newer than 5.6
    \item Old to new: 2.4dev1, 2.4a1, 2.4rc2, 2.4, 2.4-1, 2.4.1
    \item Same: 2.4-1, 2.4p1, 2.4-p1
    \item Old to new: 3.2-shared, 3.2
  \end{itemize}
\end{frame}

\section{Topics to Discuss}

\begin{frame}{Questions for users of Lmod}
  \begin{itemize}
    \item How big are your system?
    \item Any of you use a lustre based home file system?
    \item Biggest headaches with Module Systems or Lmod:
      \begin{itemize}
        \item Trouble dealing with staff?
        \item Trouble converting users?
      \end{itemize}
    \item Software Hierarchy vs Prereq/Conflict?
    \item Best story/Worst Story in dealing with Modules/Lmod?
  \end{itemize}
\end{frame}

\begin{frame}{Topics to Discuss}
  \begin{itemize}
    \item Getting Bash to work right.
    \item Leveraging Lmod to know what software your users are using.
    \item Using modulefiles and a package manager (e.g. rpm)
    \item Using SitePackage.
    \item Feature requests.
  \end{itemize}
\end{frame}

\begin{frame}{Projects}
  \begin{itemize}
    \item Fast directory tree walker for lustre. (similar
      functionality to luafilesystem) (Anyone interested?)
    \item Allow users with personal modules use system caches (R. McLay)
    \item Produce Json output of entire module system to populate
      system software web-pages (R. McLay - just completed)
    \item Support for options after commands:  \texttt{module keyword --prop mic}
    \item Other support for properties: searching?,
    \item What happens when a site wants two or more sets of properties?
  \end{itemize}
\end{frame}





\section{Conclusion}

\begin{frame}{For more information}
  \begin{itemize}
    \item Download source from: lmod.sourceforge.net (lmod.sf.net)
    \item Documentation: www.tacc.utexas.edu/tacc-projects/mclay/lmod
    \item Introduction to Lmod talk at lmod.sf.net.
    \item Best Practices Paper from SC'11
  \end{itemize}
\end{frame}

\section{Extra}


\begin{frame}[fragile]
    \frametitle{.lmodrc.lua}
  {\tiny
    \begin{alltt}
propT = \{
   arch = \{
      validT = \{ mic = 1, offload = 1, gpu = 1, \},
      displayT = \{
         ["mic:offload"]     = \{ short = "(*)",  color = "red",     doc = "...",\},
         ["mic"]             = \{ short = "(m)",  color = "blue",    doc = "...",\},
         ["offload"]         = \{ short = "(o)",  color = "blue",    doc = "...",\},
         ["gpu"]             = \{ short = "(g)",  color = "magenta", doc = "...",\},
         ["gpu:mic"]         = \{ short = "(gm)", color = "magenta", doc = "...",\},
         ["gpu:mic:offload"] = \{ short = "(@)",  color = "magenta", doc = "...",\},
      \},
   \},
\}

    \end{alltt}
}
\end{frame}

\begin{frame}{Getting Bash to work right}
  \begin{itemize}
    \item At TACC we rebuild bash so that it reads /etc/bashrc on
      interactive shells.
    \item It also reads \texttt{/etc/bash\_logout} on logout.
    \item We patch config-top.h to change bash behavior.
    \item Ubuntu does the same, Red Hat does not.
  \end{itemize}
\end{frame}

\begin{frame}{SitePackage}
  \begin{itemize}
    \item Lmod now has a standard way to include site-specific functions
    \item See Contrib/SitePackage for details
  \end{itemize}
\end{frame}


\begin{frame}{Track software usage}
  \begin{itemize}
    \item Track software usage via syslog or logout data.
    \item Lmod can build a reverse map: directories to modules
  \end{itemize}
\end{frame}

%%\begin{frame}{Topics to Discuss}
%%  \begin{itemize}
%%    \item Getting Bash to work right.
%%    \item Leveraging Lmod to know what software your users are using.
%%    \item Best Practices w.r.t. module files
%%    \item Using modulefiles and a package manager (e.g. rpm)
%%    \item Using SitePackage.
%%    \item Feature requests.
%%  \end{itemize}
%%\end{frame}




\end{document}
