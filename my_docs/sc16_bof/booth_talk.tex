\documentclass{beamer}

\usetheme[headernav]{TACC} %%Drop the 'headernav' if you don't like
                           %%the stuff at the top of your slide

\usepackage{amsmath,amssymb,amsthm}
\usepackage{alltt}
\usepackage{graphicx}

\title{SC16: 6th Annual Lmod Booth Talk}


\author{Robert McLay}
\institute{The Texas Advanced Computing Center}

\date{November 15, 2016}  %% Use this if you want to fix the date in
                          %% stone rather than use \today

\newcommand{\bfnabla}{\mbox{\boldmath$\nabla$}}
\newcommand{\laplacian}[1]{\bfnabla^2 #1}
\newcommand{\grad}[1]{\bfnabla #1}
\newcommand{\tgrad}[1]{\bfnabla^T #1}
\newcommand{\dvg}[1]{\bfnabla \cdot #1}
\newcommand{\curl}[1]{\bfnabla \times #1}
\newcommand{\lap}[1]{\bfnabla^2 #1}

\begin{document}

\begin{frame}
  \titlepage
\end{frame}

\section{Introduction}

\begin{frame}{Introduction}
  \begin{itemize}
    \item Welcome to the 6th annual TACC Booth Talk
    \item What is Lmod?
    \item Why you might to use it
    \item What is new with Lmod?
  \end{itemize}
\end{frame}

\begin{frame}{Lmod's Big Ideas}
  \begin{itemize}
    \item A modern replacement for a tried and true concept.
    \item The guiding principal: ``Make life easier w/o getting in
      the way.''
  \end{itemize}
\end{frame}

\begin{frame}{Why You Might Want To Use Lmod}
  \begin{itemize}
    \item Active Development;  Frequent Releases; Bug fixes.
    \item Vibrant Community
    \item It is used from Norway to Isreal to New Zealand from Stanford to MIT to NASA
    \item Enjoy many capabilities w/o changing a single module file
    \item Debian and Fedora packages available
    \item Many more advantages when you're ready
    \item It is what we use every day!
  \end{itemize}
\end{frame}

\begin{frame}{Features}
  \begin{itemize}
    \item Reads for TCL and Lua modulefiles
    \item One name rule.
    \item Support Software Hierarchy
    \item Spider Cache, Properties (gpu, mic)
    \item family(``compiler'') family(``mpi'') support
    \item Many other features: ml, collections, ...
  \end{itemize}
\end{frame}

\begin{frame}{History of Support for Module Names}
  \begin{itemize}
    \item Originally only N/V:  gcc/4.8.1
    \item Lmod 5+ C/N/V:  compiler/gcc/4.8.1
  \end{itemize}
\end{frame}

\begin{frame}{New with Lmod 7}
  \begin{itemize}
    \item Support for N/V/V:  fftw/64/3.3.4
    \item MODULERC Support:
      \begin{itemize}
        \item Set Defaults under Site and/or User
        \item Hide any installed module
      \end{itemize}
    \item Major refactoring of Lmod.
  \end{itemize}
\end{frame}

\begin{frame}{Setting Defaults}
  \begin{itemize}
    \item System MODULERC file: \texttt{/path/to/lmod/etc/rc}
    \item User \texttt{\textasciitilde/.modulerc}
    \item Can set defaults User, System, Files
    \item Examples: account for web services
  \end{itemize}
\end{frame}

\begin{frame}{Hiding Modules}
  \begin{itemize}
    \item System MODULERC file: \texttt{/path/to/lmod/etc/rc}
    \item User \texttt{\textasciitilde/.modulerc}
    \item \texttt{\color{blue}hide-version foo/1.2.3}
    \item Hidden from avail, spider and keyword
    \item Sites: deprecation, experimental
    \item show hidden: \texttt{module --show-hidden avail}
  \end{itemize}
\end{frame}


\begin{frame}{Conclusions: Lmod 7+}
  \begin{itemize}
    \item Latest version: github.com:TACC/Lmod.git
    \item Stable version: lmod.sf.net
    \item Documentation:  lmod.readthedocs.org
  \end{itemize}
\end{frame}


\end{document}
