\documentclass[12pt]{article}
%\usepackage{diss_inc}
%\usepackage{psfig}
\usepackage{underscore}
\setlength{\textwidth}{6.5in}
\setlength{\textheight}{9.0in}
\setlength{\topmargin}{0.0in}
\setlength{\headheight}{0.0in}
\setlength{\headsep}{0.0in}
\setlength{\oddsidemargin}{0in}
\setlength{\evensidemargin}{0in}
\begin{document}


\begin{center}
Proposal for SC-15 TACC Booth Tutorial Hands-On with Lmod \\
Robert McLay
\end{center}

Lmod is the software that ``sets the table'' for our users by allowing
them to load the software packages they need to conduct their
research.  Needs vary, and users need help selecting the correct
combination of tools from among the thousands available on TACC
systems.  Lmod, a replacement for the TCL/C based environment module
system, is part of the ``secret sauce'' that allows our users to do so
in a way that protects them from many common mistakes.  With Lmod,
users cannot load mismatched combinations of compilers, libraries, and
software tools.

In 2009 I released Lmod as sourceforge project.  Since that time the
number of site using Lmod has increased.  It is used at sites such as
NASA, MIT, Harvard, OSU, NCAR, Stanford in the U.S. It is also used
around the world including Norway, Israel. New Zealand. The Juelich
supercomputer center just installed Lmod on their latest cluster. I
moderate a very active Lmod user group, and continue to add features
and improvements in response to growing needs from sites around the
world.

The Lmod events at SC have been a success for 4 years running.  Bill
Barth conducted the first one at SC 11 and I have been giving them for
the past 3 years.  There have been 15 to 20 people (all non TACC
employees) at the recent ones.  The audience has ranged from people
who have never used Lmod to seasoned users of Lmod who want to know
about the latest features.

Given this wildly diverse audience, I propose to break up the
presentation into three parts.   The first part will be a brief
introduction to Lmod and a summary of recent changes. This will be
about 15 minutes long.  The next part will be a general question and
answer session.

The third part will be hands-on.  I will provide participants with a
virtual machine on a thumb drive that will have Lmod and some other
tools installed; participants can copy to their laptop and experiment
on their own while in the booth.  Once the presentation is over they
will be able to take the VM with them as an example of working
software.  I've successfully handed out VMs at SC 14, ISC 14, XSEDE 14
and ISC 15.

In order to entice a diverse audience, I plan to put together several
independent, optional hands-on exercises that will allow interested
participants to explore topics that interest them. Among the likely
possibilities: how to install Lmod without root permissions; how to
test an installation, how to track module usage.

I will ask a few members of the Software Tools group to assist during
this hands-on session.

\end{document}
