\documentclass{beamer}

% You can also use a 16:9 aspect ratio:
%\documentclass[aspectratio=169]{beamer}

\usetheme{TACC16}

% It's possible to move the footer to the right:
%\usetheme[rightfooter]{TACC16}


\begin{document}
\title[short version of the title]{Kind of a long title to see if this works}
\subtitle{This is a subtitle, normally not used, but just in case}

\author{Name LastName} 
\date{January 1st, 1111} 

\frame{\titlepage} 

\frame{\frametitle{Table of contents}\tableofcontents} 


\section{First Section} 

\begin{frame}{Frame title}
\begin{itemize}
 \item Item 1
 \item Item 2
 \begin{itemize}
  \item SubItem 1
  \item SubItem 2
  \begin{itemize}
  	\item SubSubItem 1
  \end{itemize}
\end{itemize}
\end{itemize}
\begin{block}{This is a block?}
Text inside the block
\end{block}
\alert{This is an alert text}
\end{frame}

\begin{frame}{Blocks \& Columns}
  \begin{columns}[T]
    \begin{column}{.5\textwidth}
      \begin{exampleblock}{Example block}
        Text
      \end{exampleblock}
    \end{column}
    \begin{column}{.5\textwidth}
      \begin{alertblock}{Alert Block}
        Text
      \end{alertblock}
    \end{column}
  \end{columns}
\end{frame}

\begin{frame}{Math}
The mass-energy equivalence is described by the famous equation
 
$$E=mc^2$$
 
discovered in 1905 by Albert Einstein. 
In natural units ($c$ = 1), the formula expresses the identity
 
\begin{equation}
E=m
\end{equation}

Extra:

$ \sum_{i=1}^{\infty} $


$ \Bigg \langle 3x+7 \Bigg \rangle $




 $ a_0+\cfrac{1}{a_1+\cfrac{1}{a_2+\cfrac{1}{a_3+\cdots}}} $
\end{frame}

\end{document}

