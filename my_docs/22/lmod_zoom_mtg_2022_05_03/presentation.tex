\documentclass{beamer}

% You can also use a 16:9 aspect ratio:
%\documentclass[aspectratio=169]{beamer}
\usetheme{TACC16}

% It's possible to move the footer to the right:
%\usetheme[rightfooter]{TACC16}

%% page 
%\begin{frame}{}
%  \begin{itemize}
%    \item
%  \end{itemize}
%\end{frame}
%
%% page 
%\begin{frame}[fragile]
%    \frametitle{}
% {\tiny
%    \begin{semiverbatim}
%    \end{semiverbatim}
%}
%  \begin{itemize}
%    \item
%  \end{itemize}
%
%\end{frame}

\begin{document}
\title[Lmod]{The complicated story about TCL break}
\author{Robert McLay} 
\date{May. 3, 2022}

% page 1
\frame{\titlepage} 


% page 2
\begin{frame}{Outline}
  \center{\includegraphics[width=.9\textwidth]{Lmod-4color@2x.png}}
  \begin{itemize}
    \item Let's talk about TCL break (and LmodBreak)
    \item Lmod didn't really support TCL break at all until Lmod 8.6
      (really Lmod 8.7)
    \item Years ago mailing list question: support for break
    \item Lmod 6 and below could not support break
    \item Why?
  \end{itemize}
\end{frame}

% page 3
\begin{frame}[fragile]
    \frametitle{Reminder: How Lmod works}
  \begin{itemize}
    \item In order to have a command effect the current shell
    \item A simple module command for bash is given below
    \item The \$LMOD\_CMD command generate shell commands as text
    \item The eval "..." evaluate the text to change the current shell
    \item For the rest of this talk: focus on what \$LMOD\_CMD produces
  \end{itemize}
 {\small
    \begin{semiverbatim}
     module () \{ eval "\$(\$LMOD\_CMD bash "\$@")"; \}
    \end{semiverbatim}
}
\end{frame}

% page 
\begin{frame}[fragile]
    \frametitle{Reminder: How Lmod TCL processing works}
  \begin{itemize}
    \item Internally Lmod knows when a file is a TCL modulefile 
    \item No *.lua extension
    \item The program tcl2lua.tcl is called to process the tcl 
    \item It converts TCL module command into Lua with Lmod module
      commands
  \end{itemize}
 {\tiny
    \begin{semiverbatim}
setenv FOO bar $\Rightarrow$ setenv("FOO","bar")
prepend-path PATH /path/to/prgm/bin $\Rightarrow$ prepend_path("PATH","/path/to/prgm/bin")
    \end{semiverbatim}
}
\end{frame}

% page 4
\begin{frame}{Reminder: How Lmod TCL processing works}
  \begin{itemize}
    \item 
  \end{itemize}
\end{frame}

% page 3
\begin{frame}{Lmod works differently than the original Tmod}
  \begin{itemize}
    \item I have never understood the TCL/C version of Tmod
    \item But I believe that it generated output
    \item The following slides outline the steps
  \end{itemize}
\end{frame}

% page 4
\begin{frame}[fragile]
    \frametitle{Create a new tclmodule test}
  \begin{itemize}
    \item Create directory rt/tclmodule/mf
    \item Create a new tcl file a/1.0
  \end{itemize}

 {\small
    \begin{semiverbatim}
#%Module

global env
set home \$env(HOME)
set pkg "\$home/foo"
prepend-path PATH \$pkg/bin
setenv FOO\_DIR \$pkg

module-whatis "This \$pkg"
    \end{semiverbatim}
}
\end{frame}

% page 5
\begin{frame}[fragile]
    \frametitle{Create a tclmodule.tdesc}
  \begin{itemize}
    \item Copy rt/nvv/nvv.tdesc to rt/tclmodule/tclmodule.tdesc
    \item Edit to fit
    \item Add steps
    \item Test tclmodules
    \item update *.txt files
    \item run all tests
    \item update *.txt files when necessary
  \end{itemize}
 {\tiny
    \begin{semiverbatim}
   runScript = [[

     . \$(projectDir)/rt/common\_funcs.sh

     unsetMT
     initStdEnvVars
     export MODULEPATH=$(testDir)/mf

     rm -fr _stderr.* _stdout.* err.* out.* .lmod.d

     runLmod --version                         # 1
     runLmod whatis a                          # 2
    \end{semiverbatim}
}
\end{frame}

% page 6
\begin{frame}[fragile]
    \frametitle{How to validate new test}
  \begin{itemize}
    \item tcl2lua.tcl mf/a/1.0
    \item There is an extra space
  \end{itemize}
 {\tiny
    \begin{semiverbatim}
tcl2lua.tcl mf/a/1.0                
prepend_path{"PATH","/Users/mclay/foo/bin",delim=":",priority="0"}
setenv("FOO_DIR","/Users/mclay/foo")
whatis([===[This /Users/mclay/foo ]===])
    \end{semiverbatim}
}
\end{frame}

% page 7
\begin{frame}[fragile]
    \frametitle{Change tcl2lua.tcl to use trimright}
  \begin{itemize}
    \item The Tcl version appends the args together w/ a space
    \item Need to trim the trailing space
  \end{itemize}
 {\tiny
    \begin{semiverbatim}
proc module-whatis \{ args \} \{
    global g\_outputA
    set msg ""
    foreach item $args \{
       append msg $item
       append msg " "
    \}

    regsub -all \{[\textbackslash{}n]\} \$msg  " " msg2
    {\color{blue}{}set msg2 [string trimright \$msg2]  # <=== add this line}
    lappend g\_outputA  "whatis(\textbackslash{}[===\textbackslash{}[\$msg2\textbackslash{}]===\textbackslash{}])\textbackslash{}n"
\}
    \end{semiverbatim}
}
\end{frame}

% page 8
\begin{frame}{Run all tests}
  \begin{itemize}
    \item Problems with tests: ck\_mtree\_syntax and softwarePage 
    \item Show diffs
    \item update *.txt gold files
  \end{itemize}
\end{frame}

% page 9
\begin{frame}[fragile]
    \frametitle{New Principal: Unload can never fail}
  \begin{itemize}
    \item A TACC colleague pointed out this problem
    \item In trying to unload a module it failed
    \item It has never been reported to me before
    \item $\Rightarrow$ module unload should \emph{never} fail
  \end{itemize}
 {\tiny
    \begin{semiverbatim}
% ml init\_opencl/2021.4.0
% unset INTELFPGAOCLSDKROOT
% ml -init\_opencl/2021.4.0
Lmod has detected the following error:
/scratch1/projects/compilers/oneapi\_2021.4.0.3422/modulefiles/init\_opencl/2021.4.0:
(init\_opencl/2021.4.0): can't read "::env(INTELFPGAOCLSDKROOT)": no such variable
    \end{semiverbatim}
}
\end{frame}

% page 10
\begin{frame}[fragile]
    \frametitle{Change LmodError when unloading}
  \begin{itemize}
    \item src/MC_Unload.lua controls what happens during unload
    \item Need to change M.error to MasterControl.warning
  \end{itemize}
\end{frame}

% page 11
\begin{frame}[fragile]
    \frametitle{Fix test unload}
  \begin{itemize}
    \item This module previously failed to unload
    \item Now it produces a warning.
  \end{itemize}

 {\small
    \begin{semiverbatim}
setenv("TOTO","set\_in\_B/2.0")
if (mode() == "unload") then
   LmodError("Error in unload of B/2.0")
end
    \end{semiverbatim}
}
\end{frame}

% page 12
\begin{frame}{Future Topics}
  \begin{itemize}
    \item What does break mean when unloading a module
    \item pushenv("CC", "gcc") in a partial software hierarchy
    \item Explain how Mname object converts names into a filename.
    \item More internals of Lmod?
  \end{itemize}
\end{frame}

\end{document}
