\documentclass{beamer}

% You can also use a 16:9 aspect ratio:
%\documentclass[aspectratio=169]{beamer}
\usetheme{TACC16}

% It's possible to move the footer to the right:
%\usetheme[rightfooter]{TACC16}

%% page 
%\begin{frame}{}
%  \begin{itemize}
%    \item
%  \end{itemize}
%\end{frame}
%
%% page 
%\begin{frame}[fragile]
%    \frametitle{}
% {\tiny
%    \begin{semiverbatim}
%    \end{semiverbatim}
%}
%  \begin{itemize}
%    \item
%  \end{itemize}
%
%\end{frame}

\begin{document}
\title[Lmod]{The complicated story about TCL break}
\author{Robert McLay} 
\date{May. 3, 2022}

% page 1
\frame{\titlepage} 


% page 2
\begin{frame}{Outline}
  \center{\includegraphics[width=.9\textwidth]{Lmod-4color@2x.png}}
  \begin{itemize}
    \item Let's talk about TCL break (and LmodBreak)
    \item Lmod didn't really support TCL break at all until Lmod 8.6
      (really Lmod 8.7)
    \item Years ago mailing list question: support for break
    \item Lmod 6 and below could not support break
    \item Why?
  \end{itemize}
\end{frame}

% page 3
\begin{frame}[fragile]
    \frametitle{Reminder: How Lmod works}
  \begin{itemize}
    \item In order to have a command effect the current shell
    \item A simple module command for bash is given below
    \item The \$LMOD\_CMD command generate shell commands as text
    \item The eval "..." evaluate the text to change the current shell
    \item For the rest of this talk: focus on what \$LMOD\_CMD produces
  \end{itemize}
 {\small
    \begin{semiverbatim}
     module () \{ eval "\$(\$LMOD\_CMD bash "\$@")"; \}
    \end{semiverbatim}
}
\end{frame}

% page 4
\begin{frame}[fragile]
    \frametitle{Reminder: How Lmod TCL processing works}
  \begin{itemize}
    \item Internally Lmod knows when a file is a TCL modulefile 
    \item No *.lua extension $\Rightarrow$ TCL modulefile 
    \item The program tcl2lua.tcl is called to process the tcl 
    \item It converts TCL modulefile into Lua with Lmod module
      commands
  \end{itemize}
 {\tiny
    \begin{semiverbatim}
setenv FOO bar \Rightarrow setenv("FOO","bar")
prepend-path PATH /prgm/bin \Rightarrow prepend_path("PATH","/prgm/bin")
break \Rightarrow LmodBreak() -- Only for bare breaks
    \end{semiverbatim}
}
\end{frame}

% page 5
\begin{frame}[fragile]
    \frametitle{TCL Break}
 {\small
   \begin{semiverbatim}
for \{set i 0\} \{\$i < 5\} \{incr i\} \{
  puts stderr "\$i"
  if \{ \$i == 3 \} \{
      break  # This breaks out of the loop
  \}
\}
break # This causes the modulefile
      # to stop being processed.       
    \end{semiverbatim}
}
\end{frame}

% page 6
\begin{frame}{Why was TCL break such a problem for Lmod?}
  \begin{itemize}
    \item TCL break stops processing the current module
    \item It ignores any changes in a module that has a break
    \item But it keeps all other modules loaded.
    \item {\color{blue} \texttt{module load A B C D}} 
    \item Where C has a break
    \item Then A B are loaded but C and D are not.
  \end{itemize}
\end{frame}

% page 7
\begin{frame}{LmodError is different}
  \begin{itemize}
    \item {\color{blue} \texttt{module load A B C D}} 
    \item Where C has an LmodError()
    \item No modules are loaded.
  \end{itemize}
\end{frame}

% page 8
\begin{frame}{Lmod waits to produce output}
  \begin{itemize}
    \item When loading several modules, Lmod waits
    \item All module actions are completed internally
    \item Then Lmod generates shell command output.
    \item Lmod 6 and earlier wouldn't know what changes to ignore when
      processing a break.
    \item Lmod produces either an error or environment changes
      not both.
  \end{itemize}
\end{frame}

% page 9
\begin{frame}{Lmod 7+ was a complete re-write of Lmod}
  \begin{itemize}
    \item It was needed to support Name/Version/Version (N/V/V) modulefiles
    \item Before Lmod only supported N/V or C/N/V
    \item Lmod 7+ now has a FrameStk (AKA the stack-frame)
    \item The FrameStk contains a stack of the environment var table (varT) and the
      module table (mt)
  \end{itemize}
\end{frame}

% page 10
\begin{frame}{FrameStk: varT and mt}
  \begin{itemize}
    \item The table varT contains key-value pairs that represent the
      new env. var values
    \item The table mt is the module table containing the currently
      loaded modules among other things
    \item The Module Table is stored in the environment via
      \$\_ModuleTable001\_ etc.
  \end{itemize}
\end{frame}

% page 11
\begin{frame}[fragile]
    \frametitle{assignment versus deepcopy() in Lua}
 {\small
    \begin{semiverbatim}
  a = \{\}
  a[1] = "foo"
  b    = a
  b[1] = "bar"
  print(a[1])  -> ``bar'' not ``foo''
    \end{semiverbatim}
}
  \begin{itemize}
    \item Lua tries to be efficent
    \item It just copies reference
    \item As shown above.
    \item Lmod provides deepcopy() function.
    \item This creates a new table
  \end{itemize}
\end{frame}

% page 12
\begin{frame}{FrameStk}
  \begin{itemize}
    \item Before each module: Deep Copy copies the previous varT and mt to top of
      FrameStk.
    \item Each evaluation of modulefile is updated on the top of the
      FrameStk
    \item When the current modulefile evaluation is completed
    \item The FrameStk is pop'ed
    \item The previous stack values are replaced with current
  \end{itemize}
\end{frame}

% page 13
\begin{frame}{FrameStk implications}
  \begin{itemize}
    \item Cannot trust local values of mt
    \item Lmod constantly has to refresh mt:
    \item {\color{blue}mt = frameStk:mt()}
    \item Because a module load might have updated it.
  \end{itemize}
\end{frame}

% page 14
\begin{frame}[fragile]
    \frametitle{LmodBreak or TCL break}
  \begin{itemize}
    \item If LmodBreak() is called, the current module changes are
      ignored
    \item LmodBreak() causes the previous values to be current
    \item FrameStk:pop() pops the stack.
    \item The FrameStk code is shown below:
  \end{itemize}
 {\tiny
    \begin{semiverbatim}
function M.LmodBreak(self)
   local stack           = self.\_\_stack
   local count           = self.\_\_count
   stack[count].mt       = deepcopy(stack[count-1].mt)
   stack[count].varT     = deepcopy(stack[count-1].varT)
end

function M.pop(self)
   local stack           = self.\_\_stack
   local count           = self.\_\_count
   stack[count-1].mt     = stack[count].mt
   stack[count-1].varT   = stack[count].varT
   stack[count]          = nil
   self.__count          = count - 1
end
    \end{semiverbatim}
}
\end{frame}

% page 15
\begin{frame}{Support for TCL break}
  \begin{itemize}
    \item Lmod 8.6+ added support LmodBreak()
    \item Lmod 8.6+ added support a bare TCL break
    \item Lmod 8.7+ added support for regular break and bare break
  \end{itemize}
\end{frame}

% page 16
\begin{frame}[fragile]
    \frametitle{TCL Break strangeness}
 {\small
   \begin{semiverbatim}
for \{set i 0\} \{\$i < 5\} \{incr i\} \{
  puts stderr "\$i"
  if \{ \$i == 3 \} \{
      break  # This breaks out of the loop
  \}
\}
break # This causes the modulefile
      # to stop being processed.       
    \end{semiverbatim}
}
  \begin{itemize}
    \item TCL treats a bare break as an error
    \item Tmod 3, 4 and 5 catch the error
    \item Lmod 8.7+ now catch the error too!
  \end{itemize}
\end{frame}

% page 17
\begin{frame}[fragile]
    \frametitle{To support regular and bare break in TCL in tcl2lua.tcl}
 {\tiny
    \begin{semiverbatim}
set sourceFailed [catch \{source \$ModulesCurrentModulefile \} errorMsg] # (1) 
set returnval 0
if \{ $g\_help && [info procs "ModulesHelp"] == "ModulesHelp" \} \{
   # handle module help
   ... 
\}
if \{\$sourceFailed\} \{
    if \{ \$sourceFailed == 3 || \$errorMsg == \{invoked "break" outside of a loop\}\} \{
       set returnVal 1
       myBreak
       showResults
       return \$returnVal
    \}
    reportError \$errorMsg
    set returnVal 1
\}
showResults
return $returnVal
    \end{semiverbatim}
}
  \begin{itemize}
    \item line 1 evaluate the TCL modulefile
    \item \$sourceFailed will be non-zero for TCL errors
    \item \$sourceFailed == 3 means a bare break has been found.
  \end{itemize}
\end{frame}

% page 18
\begin{frame}[fragile]
    \frametitle{Lmod 8.6.15 could create an endless loop}
 {\tiny
    \begin{semiverbatim}
% cat foo3/1.0
#%Module
catch \{set foo \$env(FOO)\}
if \{ [info exists foo] \} \{
  puts stderr "already set"
  break
\}
setenv FOO "just me"

% module load foo3/1.0; module load foo3/1.0
already set
already set
...continues until crtl+C ...
    \end{semiverbatim}
}
  \begin{itemize}
    \item Loading foo3/1.0 twice causes an endless loop
    \item Why?
    \item The second load forces foo3/1.0 to unload (which it can't)
    \item Lmod then tries to re-load foo3/1.0 which causes the unload etc.
  \end{itemize}

\end{frame}

% page 19
\begin{frame}{LmodBreak() is a no-op on unload}
  \begin{itemize}
    \item A bare TCL break becomes an LmodBreak() when translated.
    \item LmodBreak() does nothing during unload.
    \item This prevents the endless loop shown above. (Lmod 8.7+)
  \end{itemize}
\end{frame}

% page 20
\begin{frame}{Conclusions}
  \begin{itemize}
    \item Implementing break is trickier then you might think.
    \item Lmod now can support bare breaks finally in Lmod 8.7
    \item The FrameStk is the price to be paid to support break.
  \end{itemize}
\end{frame}

% page 21
\begin{frame}{Future Topics}
  \begin{itemize}
    \item Next Meeting: June 7th 9:30 US Central (14:30 UTC)
    \item Show how Lmod processes a module load command, stepping
      through the codebase.
    \item Suggestions.
  \end{itemize}
\end{frame}

\end{document}
