\documentclass{beamer}

% You can also use a 16:9 aspect ratio:
%\documentclass[aspectratio=169]{beamer}
\usetheme{TACC16}

% It's possible to move the footer to the right:
%\usetheme[rightfooter]{TACC16}

\begin{document}
\title[Lmod]{Settarg}
\author{Robert McLay} 
\date{Dec. 7, 2021}

% page 1
\frame{\titlepage} 

% page 2
\begin{frame}{Settarg}
  \center{\includegraphics[width=.9\textwidth]{Lmod-4color@2x.png}}
  \begin{itemize}
    \item What is settarg and how did it get started?
    \item What are the goals of settarg?
    \item What does settarg actually do?
    \item Examples
    \item Personalizing settarg: user, projects
    \item Conclusions
  \end{itemize}
\end{frame}

% page 3
\begin{frame}{What is settarg?}
  \begin{itemize}
    \item The settarg module connects the build system to:
      \begin{itemize}
        \item What modules are loaded: compiler, mpi, ...
        \item Type of build: debug/optimize/...
        \item Machine Arch: x86\_64, ppc64le, arm64, ...
      \end{itemize}
    \item Provides environment variables to connnect to the build system
    \item Makes it easy to switch between different builds
    \item Integrated with Lmod and is installed with it.
  \end{itemize}
\end{frame}

% page 4
\begin{frame}{Story: What led to settarg}
  \begin{itemize}
    \item When processor speeds where measured in MHz and not GHz
    \item Worked in a lab that had incompatible architectures with a
      shared home file system.
    \item Wanted to build programs w/o doing a {\color{blue}\texttt{make
          clean}} in-between.
    \item How about different optimization levels? (dbg, opt, mdbg?)
    \item Different compilers or MPI stacks?
  \end{itemize}
\end{frame}

% page 5
\begin{frame}{Where the settarg name came from}
  \begin{itemize}
    \item Originally wanted \$TARGET
    \item But many other projects (PETSc, etc) already use that
    \item \$TARGET shorten to \$TARG
    \item This tool was designed to set \$TARG $\Rightarrow$ {\color{blue}settarg}
  \end{itemize}
\end{frame}

% page 6
\begin{frame}{Goals of settarg}
  \begin{itemize}
    \item Make switching between DBG/OPT builds easier.
    \item Make switching between compiler/mpi and cmplr/mpi versions easier
    \item Integrate with Build process via Env. Vars.
    \item Integrate with \$PATH
    \item Optionally report status in titlebar
    \item Settarg became part of Lmod (because I can't remember to do
      two things at once!?)
  \end{itemize}
\end{frame}

% page 7
\begin{frame}{The 4 things that settarg does}
  \begin{itemize}
    \item Reads the state of the loaded modules
    \item Build \$TARG variables
    \item Change \$PATH with new \$TARG
    \item Optionally changes the titlebar
  \end{itemize}
\end{frame}

% page 8
\begin{frame}{Reasons to use settarg}
  \begin{itemize}
    \item Switching compiler can be helpful.
    \item C++ error msgs are confusing,
    \item Different compilers might make better sense.
    \item Debugging with gcc; Use intel for performance
    \item Comparing two different version of a compiler with your
      application.
  \end{itemize}
\end{frame}

% page 9
\begin{frame}{Typical \$TARG variables in dbg state}
  \begin{itemize}
    \item \$TARG $\Rightarrow$ OBJ/\_x86\_64\_dbg\_gcc-9.3.0\_mpich-3.3.2
    \item \$TARG\_BUILD\_SCENARIO $\Rightarrow$  dbg
    \item \$TARG\_COMPILER\_FAMILY $\Rightarrow$  gcc
    \item \$TARG\_MPI\_FAMILY $\Rightarrow$  mpich
    \item \$TARG\_COMPILER $\Rightarrow$  gcc-9.3.0
    \item \$TARG\_MPI $\Rightarrow$    mpich-3.3.2
  \end{itemize}
\end{frame}

% page 10
\begin{frame}{Typical \$TARG variables in opt state}
  \begin{itemize}
    \item \$TARG $\Rightarrow$  OBJ/\_x86\_64\_opt\_gcc-9.3.0\_mpich-3.3.2
    \item \$TARG\_BUILD\_SCENARIO\ $\Rightarrow$  opt
    \item \$TARG\_COMPILER\_FAMILY $\Rightarrow$  gcc
    \item \$TARG\_MPI\_FAMILY $\Rightarrow$ mpich
    \item \$TARG\_COMPILER $\Rightarrow$ gcc-9.3.0
    \item \$TARG\_MPI  $\Rightarrow$   mpich-3.3.2
  \end{itemize}
\end{frame}

% page 11
\begin{frame}{\$PATH and \$TARG}
  \begin{itemize}
    \item settarg inserts \$TARG into \$PATH.
    \item dbg $\Rightarrow$ PATH=OBJ/\_x86\_64\_dbg:$\sim$/bin:.:/usr/local/bin:/bin
    \item opt $\Rightarrow$ PATH=OBJ/\_x86\_64\_opt:$\sim$/bin:.:/usr/local/bin:/bin
  \end{itemize}
\end{frame}

% page 12
\begin{frame}{\$PATH searching can be dynamic!}
  \begin{itemize}
    \item Normally shells build a table of all exec's in path
    \item rehash can be required for new exec's
    \item But relative paths are evaluated \emph{everytime}.
  \end{itemize}
\end{frame}

% page 13
\begin{frame}{The settarg module defines the following commands}
  \begin{itemize}
    \item dbg $\Rightarrow$ debug
    \item opt $\Rightarrow$ optimize
    \item mdbg $\Rightarrow$ max debug
    \item empty $\Rightarrow$ no build scenario 
    \item cdt   $\Rightarrow$ cd \$TARG
    \item targ   $\Rightarrow$ echo \$TARG
    \item settarg --stt $\Rightarrow$ settarg state stored in
      environment
    \item settarg --report $\Rightarrow$ how settarg is configured.
  \end{itemize}
\end{frame}

% page 14
\begin{frame}{How is settarg connected to Lmod?}
  \begin{itemize}
    \item The settarg command is part of the module command.
    \item module () \{ eval \$(\$LMOD\_CMD bash "\$@") \&\&
             eval \$(\$\{LMOD\_SETTARG\_CMD:-:\} -s sh) \}
    \item Normally \$LMOD\_SETTARG\_CMD is ``:'' or ``'' so 2nd cmd is a
      no-op.
    \item With settarg loaded it becomes: \$LMOD\_DIR/settarg\_cmd
  \end{itemize}
\end{frame}

% page 15
\begin{frame}{What do dbg/opt/mdbg do?}
  \begin{itemize}
    \item They all set \$TARG\_BUILD\_SCENARIO
    \item It is up to the Makefile to interpret
    \item dbg $\Rightarrow$ CFLAGS = -g -O0
    \item opt $\Rightarrow$ CFLAGS = -O3
    \item mdbg $\Rightarrow$ CFLAGS = -g -O0 and array subscript checking
  \end{itemize}
\end{frame}

% page 16
\begin{frame}{Show examples}
  \begin{itemize}
    \item dbg/opt/mdbg $\Rightarrow$ \$TARG, \$PATH
    \item ml -impi $\Rightarrow$ \$TARG
    \item ml -intel $\Rightarrow$ \$TARG
  \end{itemize}
\end{frame}

% page 17
\begin{frame}{Example w/o Makefile changes}
  \begin{itemize}
    \item cd xalt; mkdir -p \$TARG; cdt; 
    \item ../../configure ...
    \item make install
  \end{itemize}
\end{frame}

% page 18
\begin{frame}[fragile]
  \frametitle{contrib/settarg/make\_example/Makefile.simple}
    {\tiny
\begin{semiverbatim}
ifeq (\$(TARG\_COMPILER\_FAMILY),gcc)
   CC := gcc
endif
ifeq (\$(TARG\_COMPILER\_FAMILY),intel)
   CC := icc
endif
ifneq (\$(TARG),)
  override O\_DIR := \$(TARG)/
endif
CFLAGS := -O3
ifeq (\$(TARG\_BUILD_SCENARIO),dbg)
   CFLAGS := -g -O0
endif 
EXEC := \$(O\_DIR)hello
SRC  := main.c hello.c
OBJS := \$(O\_DIR)main.o \$(O\_DIR)hello.o

all: \$(O\_DIR) \$(EXEC)
\$(O\_DIR):
        mkdir -p \$(O\_DIR)
\$(EXEC): \$(OBJS)
        \$(LINK.c) -o \$@ \$^

\$(O\_DIR)%.o : %.c
        \$(COMPILE.c) -o \$@ -c \$<

\$(O\_DIR)main.o : main.c hello.h
\$(O\_DIR)hello.o: hello.c hello.h
\end{semiverbatim}
    }
\end{frame}

% page 19
\begin{frame}{Show examples with Makefile.simple }
  \begin{itemize}
    \item dbg; make -f Makefile.simple; type hello
    \item opt; make -f Makefile.simple; type hello
    \item ml gcc; dbg; make -f Makefile.simple; type hello
  \end{itemize}
\end{frame}

% page 20
\begin{frame}{Show examples with Makefile }
  \begin{itemize}
    \item rm -rf OBJ/
    \item dbg; make -f Makefile; type hello
    \item opt; make -f Makefile; type hello
    \item ml gcc; dbg; make -f Makefile.simple; type hello
  \end{itemize}
\end{frame}

% page 21
\begin{frame}{Personal/Directory setting of settarg}
  \begin{itemize}
    \item Loading order of all settarg config files:
    \item System settarg lmod/settarg/settarg\_rc.lua
    \item $\sim$/.settarg.lua
    \item Current or parent directory .settarg.lua
    \item all are loaded with over-write of table entries.
  \end{itemize}
\end{frame}

% page 22
\begin{frame}{settarg --report}
  \begin{itemize}
    \item Reports the current state of rules for settarg
    \item Show example with a directory .settarg.lua
  \end{itemize}
\end{frame}

% page 23
\begin{frame}[fragile]
  \frametitle{a directory .settarg.lua}
    {\small
\begin{semiverbatim}
TargetList  = \{ "mach", "build\_scenario", "compiler", 
                   "mpi","solver", "file\_io "\}
\end{semiverbatim}
    }
\end{frame}

% page 24
\begin{frame}{Show example of \$TARG with above .settarg.lua}
  \begin{itemize}
    \item \$TARG 
    \item cd w/dao; $\Rightarrow$ \$TARG
    \item ml petsc; $\Rightarrow$ \$TARG
  \end{itemize}
\end{frame}

% page 25
\begin{frame}{Conclusions for Settarg modules}
  \begin{itemize}
    \item A way to switch between different kinds of builds: dbg/opt/mdbg
    \item Avoiding make clean in-between.
    \item Highly customizable for your needs.
    \item Can be made to work where you have settarg and other don't
    \item More detail: https://lmod.readthedocs.io/en/latest/310\_settarg.html
  \end{itemize}
\end{frame}

% page 26
\begin{frame}{Future Topics}
  \begin{itemize}
    \item Lmod Testing System?
    \item More internals of Lmod?
    \item collections?
    \item Guest Presentation of special issues?
  \end{itemize}
\end{frame}

\end{document}
