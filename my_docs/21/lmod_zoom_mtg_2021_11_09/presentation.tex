\documentclass{beamer}

% You can also use a 16:9 aspect ratio:
%\documentclass[aspectratio=169]{beamer}
\usetheme{TACC16}

% It's possible to move the footer to the right:
%\usetheme[rightfooter]{TACC16}

\begin{document}
\title[Lmod]{Lmod 8.6 New Features}
\author{Robert McLay} 
\date{Nov. 9, 2021}

% page 1
\frame{\titlepage} 

\section{Introduction}

% page 2
\begin{frame}{Lmod 8.6 New Features}
  \center{\includegraphics[width=.9\textwidth]{Lmod-4color@2x.png}}
  \begin{itemize}
    \item Release of Lmod 8.6
    \item New Features in Lmod web page
    \item \$LMOD\_QUARANTINE\_VARS
    \item \texttt{/etc/lmod/lmod\_config.lua} configuration file
    \item \texttt{source\_sh()} sourcing shell scripts inside a
      modulefile
    \item \texttt{LmodBreak()}: Stop processing modulefile but keep
      going!
  \end{itemize}
\end{frame}

% page 3
\begin{frame}{New Features in Lmod web page}
  \begin{itemize}
    \item https://lmod.readthedocs.io/en/latest/025_new.html
    \item A place where new features are and will continue to be announced.
  \end{itemize}
\end{frame}

% page 4
\begin{frame}{\$LMOD\_QUARANTINE\_VARS}
  \begin{itemize}
    \item A module at TACC turn-off  \$LMOD\_PAGER
    \item This \!\@\#\%\& module made me mad.
    \item Tmod has a new feature kinda like this.
    \item \$LMOD\_QUARANTINE\_VARS was invented.
    \item export LMOD\_QUARANTINE\_VARS=LMOD\_PAGER:LMOD\_REDIRECT
    \item This means that a module can't change those variables.
    \item This only works with regular env. vars.
    \item You can't quarantine PATH like variables.
    \item A user sets this variable in their ~/.bashrc or similar
      file.
    \item This obviously won't work for modules loaded during the
      processing of /etc/profile.d/*.sh files
  \end{itemize}
\end{frame}

% page 5
\begin{frame}{/etc/lmod/lmod\_config.lua configuration file}
  \begin{itemize}
    \item This file is evaluated during Lmod startup. 
    \item This location is the default during configuration.
    \item A site can change this location.
  \end{itemize}
    {\tiny
\begin{semiverbatim}
    require("strict")
    local cosmic       = require("Cosmic"):singleton()

    cosmic:assign("LMOD_SITE_NAME",   "XYZZY")
    local function foo()
      ...
    end
    sandbox\_registration {
       foo = foo,
    }
\end{semiverbatim}
}
\end{frame}

% page 6
%\begin{frame}[fragile]
%  \frametitle{Example Modulefile: phdf5}
%    {\tiny
%\begin{semiverbatim}
%
%setenv("TACC\_HDF5\_DIR","/apps/.../phdf5/1.12.1")
%setenv("TACC\_HDF5\_INC","/apps/.../phdf5/1.12.1/inc")
%setenv("TACC\_HDF5\_INC","/apps/.../phdf5/1.12.1/inc")
%setenv("TACC\_HDF5\_LIB","/apps/.../phdf5/1.12.1/lib")
%prepend\_path("PATH","/apps/.../phdf5/1.10.4/bin")
%prepend\_path("LD\_LIB\_PATH","/apps/.../phdf5/1.10.4/lib")
%help([[Help Message for Parallel HDF5 ...]])
%\end{semiverbatim}
%    }
%\end{frame}
%
%% page 7
%\begin{frame}[fragile]
%  \frametitle{\texttt{Bash: Module load phdf5}}
%    {\small
%\begin{semiverbatim}
%export TACC\_HDF5\_DIR=/apps/.../phdf5/1.12.1
%export TACC\_HDF5\_INC=/apps/.../phdf5/1.12.1/inc
%export TACC\_HDF5\_LIB=/apps/.../phdf5/1.12.1/lib
%export PATH=/apps/.../phdf5/1.10.4/bin:/usr/bin:/bin
%export LD\_LIB\_PATH=/apps/.../phdf5/1.10.4/lib:...
%\end{semiverbatim}
%    }
%\end{frame}
%
%% page 8
%\begin{frame}[fragile]
%  \frametitle{\texttt{Bash: Module unload phdf5}}
%    {\small
%\begin{semiverbatim}
%unset TACC\_HDF5\_DIR
%unset TACC\_HDF5\_INC
%unset TACC\_HDF5\_LIB
%export PATH=/usr/bin:/bin
%export LD\_LIB\_PATH=...
%\end{semiverbatim}
%    }
%\end{frame}
%
%% page 9
%\begin{frame}{Lua Object Oriented Programing}
%  \begin{itemize}
%    \item Lua's OOP model is OOP lite
%    \item It is simplier than Python's
%    \item There is very little Magic
%    \item It is an extension of Lua's Hash Tables (AKA Dictionaries)
%    \item Functions are First Class Object
%    \item They can be assigned to variables
%  \end{itemize}
%\end{frame}
%
%% page 10
%\begin{frame}{How does Lmod evaluate modulefile functions etc}
%  \begin{itemize}
%    \item Note that TCL modules are converted to Lua automatically
%    \item Each module function calls Lua functions (like setenv() )
%    \item Inside each function dynamically calls the correct operation
%      for load, unload, etc.
%  \end{itemize}
%\end{frame}
%
%% page 11
%\begin{frame}{Lmod finds and reads phdf5/1.12.1.lua}
%  \begin{itemize}
%    \item loadModuleFile.lua reads modulefile into a string \emph{whole}
%    \item \texttt{status, msg = sandbox\_run(whole)}
%    \item Each line in sandbox is evaluated by the lua interpreter
%  \end{itemize}
%\end{frame}
%
%% page 12
%\begin{frame}{How does Lmod handle setenv()?}
%  \begin{itemize}
%    \item Lmod could have check the mode() in each function
%    \item Instead Lmod builds MasterControl object (mcp) based on mode()
%    \item There is a derived class for Load, Unload, Show etc.
%  \end{itemize}
%\end{frame}
%
%% page 13
%\begin{frame}[fragile]
%  \frametitle{How does Lmod handle setenv()?}
%    {\tiny
%\begin{semiverbatim}
%--src/modfunc.lua
%function setenv(...)
%    -- check args
%    mcp:setenv(...)
%end
%
%--src/MasterControl.lua
%function M.setenv(self, name, value)
%   local frameStk = FrameStk:singleton()
%   local varT     = frameStk:varT()
%   if (varT[name] == nil) then
%      varT[name] = Var:new(name)
%   end
%   varT[name]:set(tostring(value))
%end
%
%function M.unsetenv(self, name, value)
%   local frameStk  = FrameStk:singleton()
%   local varT      = frameStk:varT()
%   if (varT[name] == nil) then
%      varT[name]   = Var:new(name)
%   end
%   varT[name]:unset()
%end
%\end{semiverbatim}
%    }
%\end{frame}
%
%% page 14
%\begin{frame}[fragile]
%  \frametitle{What is mcp? How does load work?}
%    {\tiny
%\begin{semiverbatim}
%--src/lmod.in.lua
%   MCP = MasterControl.build("load")
%   mcp = MasterControl.build("load")
%
%--src/cmdfuncs.lua
%function Load\_Usr(...)
%   local mcp\_old = mcp
%   mcp = MCP
%   mcp:load\_usr(...)
%   mcp = mcp\_old
%end
%
%--src/MC\_Load.lua
%...
%M.setenv               = MasterControl.setenv
%\end{semiverbatim}
%    }
%\end{frame}
%
%% page 15
%\begin{frame}[fragile]
%  \frametitle{How does unload work?}
%    {\tiny
%\begin{semiverbatim}
%--src/cmdfuncs.lua
%function UnLoad(...)
%   local mcp\_old = mcp
%   mcp = MasterControl.build("unload")
%   MCP:unload\_usr(...)
%   mcp = mcp\_old
%end
%
%--src/MC_Unload.lua
%...
%M.setenv               = MasterControl.unsetenv
%\end{semiverbatim}
%    }
%\end{frame}
%
%% page 16
%\begin{frame}{Other internal Lmod Topics}
%  \begin{itemize}
%    \item Rules for finding modulefiles for load?
%    \item The MName object?
%    \item The ModuleTable stored in the Environment
%    \item sandbox() ?
%    \item FrameStk?
%    \item tcl to lua translation?
%    \item Lmod testing system? 
%  \end{itemize}
%\end{frame}

\end{document}
