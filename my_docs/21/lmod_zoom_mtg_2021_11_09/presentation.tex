\documentclass{beamer}

% You can also use a 16:9 aspect ratio:
%\documentclass[aspectratio=169]{beamer}
\usetheme{TACC16}

% It's possible to move the footer to the right:
%\usetheme[rightfooter]{TACC16}

\begin{document}
\title[Lmod]{Lmod 8.6 New Features}
\author{Robert McLay} 
\date{Nov. 9, 2021}

% page 1
\frame{\titlepage} 

\section{Introduction}

% page 2
\begin{frame}{Lmod 8.6 New Features}
  \center{\includegraphics[width=.9\textwidth]{Lmod-4color@2x.png}}
  \begin{itemize}
    \item Release of Lmod 8.6
    \item New Features in Lmod web page
    \item \$LMOD\_QUARANTINE\_VARS
    \item \texttt{/etc/lmod/lmod\_config.lua} configuration file
    \item \texttt{source\_sh()} sourcing shell scripts inside a
      modulefile
    \item \texttt{LmodBreak()}: Stop processing modulefile but keep
      going!
  \end{itemize}
\end{frame}

% page 3
\begin{frame}{New Features in Lmod web page}
  \begin{itemize}
    \item https://lmod.readthedocs.io/en/latest/025\_new.html
    \item A place where new features are and will continue to be announced.
  \end{itemize}
\end{frame}

% page 4
\begin{frame}{\$LMOD\_QUARANTINE\_VARS}
  \begin{itemize}
    \item A module at TACC turn-off  \$LMOD\_PAGER
    \item This \!\@\#\%\& module made me mad.
    \item Tmod has a new feature kinda like this.
    \item \$LMOD\_QUARANTINE\_VARS was invented.
  \end{itemize}
\end{frame}

% page 5
\begin{frame}{\$LMOD\_QUARANTINE\_VARS (II)}
  \begin{itemize}
    \item export LMOD\_QUARANTINE\_VARS=LMOD\_PAGER:LMOD\_REDIRECT
    \item This means that a module can't change those variables.
    \item This only works with regular env. vars.
    \item You can't quarantine PATH like variables.
    \item A user sets this variable in their ~/.bashrc or similar
      file.
    \item This obviously won't work for modules loaded during the
      processing of /etc/profile.d/*.sh files
  \end{itemize}
\end{frame}

% page 6
\begin{frame}[fragile]
  \frametitle{/etc/lmod/lmod\_config.lua configuration file}
  \begin{itemize}
    \item This file is evaluated during Lmod startup. 
    \item This location is the default during configuration.
    \item A site can change this location at configuration.
  \end{itemize}
    {\small
\begin{semiverbatim}
-- Example /etc/lmod/lmod\_config.lua
require("strict")
local cosmic = require("Cosmic"):singleton()

cosmic:assign("LMOD\_SITE\_NAME", "XYZZY")
local function foo()
  ...
end
sandbox\_registration \{ foo = foo \}
\end{semiverbatim}
}
\end{frame}

% page 7
\begin{frame}{Sourcing shell scripts inside a modulefile w/ source\_sh()}
  \begin{itemize}
    \item This was implemented in Tmod 4.7
    \item Xavier told me that he did this during Covid Lockdown in France.
    \item Lmod 8.6 re-implements this feature in a similar way.
    \item It knows about env. vars and shell functions and aliases.
  \end{itemize}
\end{frame}

% page 8
\begin{frame}{source\_sh() Implementation}
  \begin{itemize}
    \item It captures the env. vars/functions/alias before and after
      the running the shell script.
    \item It computes the difference and saves it into the ModuleTable
      in env.
    \item It can be safely unloaded, shown.
    \item script path and arguments must not change between load and unload.
    \item \texttt{module refresh} works
    \item Obvious points:
      \begin{itemize}
        \item It is better to use sh\_to\_modulefile and convert once.
        \item But sh\_to\_modulefile is not dynamic (e.g. \$HOME)
        \item Can't have run the script in the user environment before
          loading the script.
      \end{itemize}
  \end{itemize}
\end{frame}

% page 9
\begin{frame}[fragile]
  \frametitle{ml --mt}
    {\tiny
\begin{semiverbatim}
\_ModuleTable\_ = \{
  MTversion = 3,
  mT = \{
    wrapperSh = \{
      fn = "/Users/mclay/w/lmod/rt/sh_to_modulefile/mf/wrapperSh/1.0.lua",
      fullName = "wrapperSh/1.0",
      loadOrder = 1,
      mcmdT = {
        ["/home/user/w/lmod/rt/sh_to_modulefile/second.sh arg1"] = \{
          "setenv(\\"SECOND\",\\"FOO_BAR\\")",
        \},
        ["/Users/mclay/w/lmod/rt/sh_to_modulefile/tstScript.sh"] = \{
          "setenv(\\"MY_NAME\",\\"tstScript.sh\\")",
          "prepend_path(\\"PATH\\",\\"/Users/mclay/w/lmod/rt/sh_to_modulefile/bin\\")",
          "set_alias(\\"fooAlias\\",\\"foobin -q -l\\")"
          , [[set_shell_function("banner"," \\
    local str=\\"$1\\";\\
    local RED='\\27[1;31m';\\
    local NONE='\\27[0m';\\
    echo \\"${RED}${str}${NONE}\\"\\
")]], 
        \},
      \},
    \},
  \},
}
\end{semiverbatim}
    }
\end{frame}

% page 10
\begin{frame}{LmodBreak()}
  \begin{itemize}
    \item Tmod3 had a function called \texttt{break}
    \item It stop processing current module
    \item But it kept going otherwise.
    \item So it is different than erroring out.
    \item Lmod: If there is an error, no modules are loaded.
  \end{itemize}
\end{frame}

% page 11
\begin{frame}{LmodBreak() (II)}
  \begin{itemize}
    \item Lmod 8.6 adds \texttt{LmodBreak(msg)} function. 
    \item TCL Modules use \texttt{break msg}
    \item No changes to environment from current module are kept.
    \item All other changes are kept.
  \end{itemize}
\end{frame}

% page 12
\begin{frame}[fragile]
  \frametitle{LmodBreak() (III)}
    {\tiny
\begin{semiverbatim}
--Stdenv.lua
load(A)
load(BRK)
load(C)

--A/1.0.lua
setenv("A","1.0")

--BRK/1.0.lua
setenv("BRK","1.0")
LmodBreak()

--C/1.0.lua
setenv("C","1.0")
\end{semiverbatim}
    }
\end{frame}

% page 13
\begin{frame}[fragile]
  \frametitle{ml StdEnv}
    {\tiny
\begin{semiverbatim}
    export A=1.0
    export C=1.0
\end{semiverbatim}
    }
\end{frame}

% page 14
\begin{frame}{Conclusions}
  \begin{itemize}
    \item Many new features added to Lmod 8.6
    \item \$LMOD\_QUARANTINE\_VARS
    \item \texttt{/etc/lmod/lmod\_config.lua} configuration file
    \item \texttt{source\_sh()} 
    \item \texttt{LmodBreak()}: 
  \end{itemize}
\end{frame}

% page 15
\begin{frame}{Future Topics}
  \begin{itemize}
    \item Lmod Testing System?
    \item More internals of Lmod?
    \item Settarg?
    \item collections?
    \item Guest Presentation of special issues?
  \end{itemize}
\end{frame}

\end{document}
