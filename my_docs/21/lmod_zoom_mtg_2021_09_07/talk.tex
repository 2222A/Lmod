\documentclass{beamer}

% You can also use a 16:9 aspect ratio:
%\documentclass[aspectratio=169]{beamer}
\usetheme{TACC16}

% It's possible to move the footer to the right:
%\usetheme[rightfooter]{TACC16}

\begin{document}
\title[Lmod]{Lmod hooks}
\author{Robert McLay} 
\date{Sept. 7, 2021}

% page 1
\frame{\titlepage} 

\section{Introduction}

% page 2
\begin{frame}{Lmod Hooks Discussion}
  \center{\includegraphics[width=.9\textwidth]{Lmod-4color@2x.png}}
  \begin{itemize}
    \item History: How to track module usage?
    \item How do hooks work?
    \item What Hooks are available?
  \end{itemize}
\end{frame}

% page 3
\begin{frame}{How to track module usage?}
  \begin{itemize}
    \item No extra code in modulefiles.
    \item Must work with modulefiles written in Lua or TCL.
    \item Must not be built-in to Lmod.
    \item $\Rightarrow$ Hooks
  \end{itemize}
\end{frame}

% page 4
\begin{frame}{Hooks}
  \begin{itemize}
    \item I am an emacs user 
    \item Is there any other editor worth using?
    \item Emacs has hooks to modify behavior
    \item Hooks and SitePackage.lua was born.
  \end{itemize}
\end{frame}

% page 5
\begin{frame}{SitePackage.lua}
  \begin{itemize}
    \item SitePackage.lua is ``sourced'' on every module command.
    \item It provides functions available for every Lua module file.
    \item Site should set \$LMOD\_PACKAGE\_PATH to the directory
      containing it.
    \item See sandbox\_registration\{\} to see how to add your own functions.
  \end{itemize}
\end{frame}

% page 6
\begin{frame}{Hooks}
  \begin{itemize}
    \item Functions name can be stored in variable, arrays, etc.
    \item In particular, Lua can store strings and a matching function
    \item Lmod has default implementations that do nothing
    \item Site can override hooks with their own implementation.
  \end{itemize}
\end{frame}

% page 7
\begin{frame}[fragile]
  \frametitle{Hooks in SitePackage.lua}
    {\small
\begin{semiverbatim}
local function load_hook(t)
   -- ...
   -- ...
end

local hook = require("hook")
hook.register("load", load_hook)
\end{semiverbatim}
    }
\end{frame}


% page 8
\begin{frame}{What can a site do with a load hook?}
  \begin{itemize}
    \item The load hooks is called on every load.
    \item TACC uses the load hook to track module usage
    \item Many EasyBuild sites track only ``user'' loads 
    \item Compute Canada track usage, set properties, and much more
  \end{itemize}
\end{frame}

% page 9
\begin{frame}[fragile]
  \frametitle{Contrib/more\_hooks/SitePackage.lua}
    {\small
\begin{semiverbatim}
  local frameStk = FrameStk:singleton()
  local userload = (frameStk:atTop()) and
                   "yes" or "no"

  local logTbl       = \{\}
  logTbl[\#logTbl+1]  = \{"userload", userload\}
  logTbl[\#logTbl+1]  = \{"module", t.modFullName\}
  logTbl[\#logTbl+1]  = \{"fn", t.fn\}

  logmsg(logTbl)
\end{semiverbatim}
    }
\end{frame}

% page 10
\begin{frame}[fragile]
  \frametitle{Compute Canada}
  \begin{itemize}
    \item https://github.com/ComputeCanada/software-stack-config/blob/main/lmod/SitePackage.lua\#L261-L272
    {\small
\begin{semiverbatim}
     local function load\_hook(t)
        local valid = validate\_license(t)
        set\_props(t)
        set\_family(t)
        default\_module\_change\_warning(t)
        log\_module\_load(t,true)
        set\_local\_paths(t)
     end
\end{semiverbatim}
    }
    \end{itemize}
\end{frame}

% page 11
\begin{frame}{What hooks are available?}
  \begin{itemize}
    \item Basic hooks
    \item Shared home Filesystem hook
    \item Advanced Hooks
  \end{itemize}
\end{frame}

% page 12
\begin{frame}{Basic hooks}
  \begin{itemize}
    \item load hook - Called on every load
    \item avail - Map directories into labels
    \item startup - called at startup
    \item finalize - called just before Lmod is about to finish
    \item isVisibleHook - Reports whether a module should be visible
      to avail and spider.
    \item SiteName - Hook to specify Site Name
      (\emph{site}\_FAMILY\_\emph{something}) for family prefix for
      hierarchical module layout: TACC\_FAMILY\_COMPILER
  \end{itemize}
\end{frame}

% page 13
\begin{frame}{isVisibleHook}
  \begin{itemize}
    \item isVisibleHook(modT): Reports whether a module should be visible
      to avail and spider.
    \item modT=\{fullName=..., sn=..., fn=...\}
    \item Where fullName is name/version, sn = name and fn is fileName
  \end{itemize}
\end{frame}

% page 14
\begin{frame}{Shared Home Filesystem hook}
  \begin{itemize}
    \item groupName - This hook adds the arch and os name to
      moduleT.lua to make it safe on shared filesystems.
  \end{itemize}
\end{frame}

% page 15
\begin{frame}{Advanced Hooks}
  \begin{itemize}
    \item unload - Called on every unload
    \item msgHook - Hook to print message after avail, list, spider
    \item errWarnMsgHook - Hook to print messages after LmodError LmodWarning, LmodMessage    
    \item restore - Hook to run after a restore operation.
    \item load\_spider - This hook is run for evaluating modules for spider/avail.
    \item listHook - This hook gets the list of active modules
    \item spider\_decoration -  This hook adds decoration to spider
      level one output. It can be the category or a property.

  \end{itemize}
\end{frame}

% page 16
\begin{frame}{HookArray}
  \begin{itemize}
    \item A module can load other modules
    \item Module X loads modules A, B, and C
    \item Suppose that A, B load and C errors out.
    \item Use load\_hook() to save loads in an array
    \item Use the HookArray to report loads via syslog.
  \end{itemize}
\end{frame}

% page 17
\begin{frame}[fragile]
  \frametitle{Contrib/TACC/SitePackage.lua: }
    {\tiny
\begin{semiverbatim}
local s\_msgA = \{\}
local function load_hook(t)
   -- the arg t is a table:
   --     t.modFullName:  the module full name: (i.e: gcc/4.7.2)
   --     t.fn:           The file name: (i.e /apps/mfiles/Core/gcc/4.7.2.lua)
   if (mode() ~= "load") then return end
   local currentTime = epoch()
   local msg         = string.format("user=%s module=%s path=%s host=%s time=%f",
                                     user, t.modFullName, t.fn, 
                                     host or "<unknown_syshost>", currentTime)
   local a           = s\_msgA
   a[#a+1]           = msg
   dbg.fini()
end
local function report_loads()
   local a = s\_msgA
   for i = 1,#a do
      local msg = a[i]
      lmod_system_execute("logger -t ModuleUsageTracking -p local0.info " .. msg)
   end
end

ExitHookA.register(report_loads)
hook.register("load",           load_hook)
\end{semiverbatim}
    }
\end{frame}


% page 18
\begin{frame}{Other hooks in the documentation}
  \begin{itemize}
    \item Avail: https://lmod.readthedocs.io/en/latest/200\_avail\_custom.html
    \item msgHook: https://lmod.readthedocs.io/en/latest/170\_hooks.html
  \end{itemize}
\end{frame}

\end{document}
