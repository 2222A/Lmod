\documentclass{beamer}

% You can also use a 16:9 aspect ratio:
%\documentclass[aspectratio=169]{beamer}
\usetheme{TACC16}

% It's possible to move the footer to the right:
%\usetheme[rightfooter]{TACC16}

\begin{document}
\title[Lmod]{Lmod Debugging \& Module evaluation}
\author{Robert McLay} 
\date{Sept. 7, 2021}

% page 1
\frame{\titlepage} 

\section{Introduction}

% page 2
\begin{frame}{Debugging Modulefiles \& How are Modulefiles evaluated? }
  \center{\includegraphics[width=.9\textwidth]{Lmod-4color@2x.png}}
  \begin{itemize}
    \item Debugging Modulefiles Tricks
    \item How are Modulefile evaluated?
  \end{itemize}
\end{frame}

% page 3
\begin{frame}{Debugging Modulefiles Tricks}
  \begin{itemize}
    \item https://lmod.readthedocs.io/en/latest/160\_debugging\_modulefiles.html
    \item Trick 1: Printing: \texttt{\$LMOD\_CMD bash load} \emph{module}
    \item Trick 2: Tracing:  \texttt{ml -T } \emph{module}
  \end{itemize}
\end{frame}

% page 4
\begin{frame}{How many ways are Modules evaluated?}
  \begin{itemize}
    \item There are 3 main ways: load, unload, show
    \item There are 10 total ways (src/MC\_*)
    \item How does Lmod handle this mess?
  \end{itemize}
\end{frame}

% page 5
\begin{frame}{The simple goals of an Env. Module System}
  \begin{itemize}
    \item Change User's environment
    \item One text file that's independent of shell (bash, zsh, csh, ...)
    \item Instead of separate shell scripts for each shell (like intel scripts)
    \item Great Feature: Unloading a module restores User's env. (kinda?!)
  \end{itemize}
\end{frame}

% page 6
\begin{frame}[fragile]
  \frametitle{Example Modulefile: phdf5}
    {\tiny
\begin{semiverbatim}

setenv("TACC\_HDF5\_DIR","/apps/.../phdf5/1.12.1")
setenv("TACC\_HDF5\_INC","/apps/.../phdf5/1.12.1/inc")
setenv("TACC\_HDF5\_INC","/apps/.../phdf5/1.12.1/inc")
setenv("TACC\_HDF5\_LIB","/apps/.../phdf5/1.12.1/lib")
prepend\_path("PATH","/apps/.../phdf5/1.10.4/bin")
prepend\_path("LD\_LIB\_PATH","/apps/.../phdf5/1.10.4/lib")
help([[Help Message for Parallel HDF5 ...]])
\end{semiverbatim}
    }
\end{frame}

% page 7
\begin{frame}[fragile]
  \frametitle{\texttt{Bash: Module load phdf5}}
    {\small
\begin{semiverbatim}
export TACC\_HDF5\_DIR=/apps/.../phdf5/1.12.1
export TACC\_HDF5\_INC=/apps/.../phdf5/1.12.1/inc
export TACC\_HDF5\_LIB=/apps/.../phdf5/1.12.1/lib
export PATH=/apps/.../phdf5/1.10.4/bin:/usr/bin:/bin
export LD\_LIB\_PATH=/apps/.../phdf5/1.10.4/lib:...
\end{semiverbatim}
    }
\end{frame}

% page 8
\begin{frame}[fragile]
  \frametitle{\texttt{Bash: Module unload phdf5}}
    {\small
\begin{semiverbatim}
unset TACC\_HDF5\_DIR
unset TACC\_HDF5\_INC
unset TACC\_HDF5\_LIB
export PATH=/usr/bin:/bin
export LD\_LIB\_PATH=...
\end{semiverbatim}
    }
\end{frame}

% page 9
\begin{frame}{Lua Object Oriented Programing}
  \begin{itemize}
    \item Lua's OOP model is OOP lite
    \item It is simplier than Python's
    \item There is very little Magic
    \item It is an extension of Lua's Hash Tables (AKA Dictionaries)
    \item Functions are First Class Object
    \item They can be assigned to variables
  \end{itemize}
\end{frame}

% page 10
\begin{frame}{How does Lmod evaluate modulefile functions etc}
  \begin{itemize}
    \item Note that TCL modules are converted to Lua automatically
    \item Each module function calls Lua functions (like setenv() )
    \item Inside each function dynamically calls the correct operation
      for load, unload, etc.
  \end{itemize}
\end{frame}

% page 11
\begin{frame}{Lmod finds and reads phdf5/1.12.1.lua}
  \begin{itemize}
    \item loadModuleFile.lua reads modulefile into a string \emph{whole}
    \item \texttt{status, msg = sandbox\_run(whole)}
    \item Each line in sandbox is evaluated by the lua interpreter
  \end{itemize}
\end{frame}

% page 12
\begin{frame}{How does Lmod handle setenv()?}
  \begin{itemize}
    \item Lmod could have check the mode() in each function
    \item Instead Lmod builds MasterControl object (mcp) based on mode()
    \item There is a derived class for Load, Unload, Show etc.
  \end{itemize}
\end{frame}

% page 13
\begin{frame}[fragile]
  \frametitle{How does Lmod handle setenv()?}
    {\tiny
\begin{semiverbatim}
--src/modfunc.lua
function setenv(...)
    -- check args
    mcp:setenv(...)
end

--src/MasterControl.lua
function M.setenv(self, name, value)
   local frameStk = FrameStk:singleton()
   local varT     = frameStk:varT()
   if (varT[name] == nil) then
      varT[name] = Var:new(name)
   end
   varT[name]:set(tostring(value))
end

function M.unsetenv(self, name, value)
   local frameStk  = FrameStk:singleton()
   local varT      = frameStk:varT()
   if (varT[name] == nil) then
      varT[name]   = Var:new(name)
   end
   varT[name]:unset()
end
\end{semiverbatim}
    }
\end{frame}

% page 14
\begin{frame}[fragile]
  \frametitle{What is mcp? How does load work?}
    {\tiny
\begin{semiverbatim}
--src/lmod.in.lua
   MCP = MasterControl.build("load")
   mcp = MasterControl.build("load")

--src/cmdfuncs.lua
function Load\_Usr(...)
   local mcp\_old = mcp
   mcp = MCP
   mcp:load\_usr(...)
   mcp = mcp\_old
end

--src/MC\_Load.lua
...
M.setenv               = MasterControl.setenv
\end{semiverbatim}
    }
\end{frame}

% page 15
\begin{frame}[fragile]
  \frametitle{How does unload work?}
    {\tiny
\begin{semiverbatim}
--src/cmdfuncs.lua
function UnLoad(...)
   local mcp\_old = mcp
   mcp = MasterControl.build("unload")
   mcp:unload(...)
   mcp = mcp\_old
end

--src/MC_Unload.lua
...
M.setenv               = MasterControl.unsetenv
\end{semiverbatim}
    }
\end{frame}

% page 16
\begin{frame}{Other internal Lmod Topics}
  \begin{itemize}
    \item Rules for finding modulefiles for load?
    \item The MName object?
    \item The ModuleTable stored in the Environment
    \item The Sandbox
  \end{itemize}
\end{frame}




\end{document}
