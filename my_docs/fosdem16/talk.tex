%* installation (deps, build/install procedure, making module/ml work)
%* configuration (either via ./configure at build time, or via 
%   $LMOD_FOO), most important configure options
%* setting up the system spider cache: configuring Lmod, updating spider 
%  cache, difference between system/user spider cache
%* + all (other) aspects mentioned on slide #26 in this slide deck: 
%   explained + example :)
%* what settarg is all about

\documentclass{beamer}

\usetheme[headernav]{TACC} %%Drop the 'headernav' if you don't like
                           %%the stuff at the top of your slide

\usepackage{amsmath,amssymb,amsthm}
\usepackage{alltt}
\usepackage{graphicx}

\title{Lmod: Building a Community around an Environment Modules Tool}


\author{Robert McLay}
\institute{The Texas Advanced Computing Center}

\date{Jan. 30, 2015}  %% Use this if you want to fix the date in
                      %% stone rather than use \today

\newcommand{\bfnabla}{\mbox{\boldmath$\nabla$}}
\newcommand{\laplacian}[1]{\bfnabla^2 #1}
\newcommand{\grad}[1]{\bfnabla #1}
\newcommand{\tgrad}[1]{\bfnabla^T #1}
\newcommand{\dvg}[1]{\bfnabla \cdot #1}
\newcommand{\curl}[1]{\bfnabla \times #1}
\newcommand{\lap}[1]{\bfnabla^2 #1}

\begin{document}

\begin{frame}
  \titlepage
\end{frame}

% On today's supercomputers, chemists, biologists, physicists and
% engineers are sharing the same system and they all need different
% software. Environment Modules have been the way since the '90 that
% users select the software they need.  They allow users to load and
% unload the packages they want.  They get to control which version of
% the software they use, rather than the system administrators.  Lmod,
% implemented in Lua, is a modern replacement for the venerable TCL/C
% based tool. Lmod offers many features to handle the vastly more
% dynamic software environment than the original tool was designed to
% handle.

% Lmod is a modern, flexible, robust environment module system: the
% system tool that "sets the table" by helping supercomputer users load
% the software packages they need to do their work. They get to control which version of
% the software they use, rather than the system administrators.  It is a Lua-based
% drop-in replacement for the venerable TCL/C environment module
% system. Because needs vary across the diverse computational research
% community, module systems like Lmod are mission-critical, ubiquitous,
% and expected on such systems: users need help selecting the correct
% combination of tools from among the thousands available on modern,
% large-scale systems. 
% 
% This is not a talk about Lmod's features or design; instead, the focus
% is on process, community, and culture.  What is it like to develop,
% maintain, and support a tool for a community that is often unaware
% that it (or Lua) even exists?  How does one go about diagnosing and
% correcting problems on systems to which you cannot get access? 
%  
% But the talk also addresses an important Lua-related issue. A module
% system loads a given software package by interpreting a 'modulefile'
% that describes the details of the environment that software package
% needs. Lmod's native modulefiles are Lua programs; Lmod parses legacy
% TCL modules by translating their instructions into Lua. This approach
% has been very successful, but has not been without its challenges:
% this talk includes an introduction to some of the associated technical
% issues and edge cases. 
%  
% This has been an interesting ride, from a local tool deployed at a
% single site in Texas, to an open source mainstay that is mission
% critical at hundreds of supercomputer centers around the world. World
% domination through Lua and Lmod, one center at a time! 

\begin{frame}{Outline}
  \begin{itemize}
    \item What are Environment Modules and Lmod?
    \item Using Lua to implement Lmod features
    \item My experience building a community
    \item Lmod tech. solutions to built trust
    \item Conclusions
  \end{itemize}
\end{frame}



\section{Environment Modules}

\begin{frame}{What are Environment Modules?}
  \begin{itemize}
    \item A tool to set (and unset!) environment variables.
    \item Useful for adding elements to \$PATH, \$LD\_LIBRARY\_PATH
    \item Also remove elements from \$PATH, etc as well.
    \item 1st paper on Env. Modules in 1991 by Furlani.
    \item There is a TCL/C based tool (a.k.a Tmod)
  \end{itemize}
\end{frame}

\begin{frame}{Why is this useful?}
  \begin{itemize}
    \item High Performance Computers have hundreds of users.
    \item Physicists, Chemists, Biologist, Engineers need different
      software
    \item Modules provide a convenient way to support them all.
    \item A software developers delight: 
      \begin{itemize}
        \item Switch compilers easily.
        \item For both versions and programs.
      \end{itemize}
  \end{itemize}
\end{frame}

\begin{frame}[fragile]
    \frametitle{An Example}
    {\small
      \begin{alltt}
          \$ {\color{blue} ddt}
          command not found: ddt
          \$ {\color{blue} module load ddt}
          \$ {\color{blue} ddt }
          \$ {\color{blue} module rm ddt}
          \$ {\color{blue} ddt}
          command not found: ddt
      \end{alltt}
    }
\end{frame}    

\begin{frame}{What is Lmod?}
  \begin{itemize}
    \item A Lua based Enviroment Module Tool that supports TCL modulefiles
    \item Tmod doesn't support a software hierarchy
    \item The C++ Boost library needs to match Compiler and Version
    \item Switch compilers should swap boost to match
    \item Lmod does this automatically
  \end{itemize}
\end{frame}

\begin{frame}{Why is this possible?}
  \begin{itemize}
    \item How can a command effect the current environment?
    \item In Unix, the child process inherits the environment
    \item Not the other way around
  \end{itemize}
\end{frame}

\begin{frame}{The trick}
  \begin{itemize}
    \item All \emph{lmod} does is produce text.
    \item \texttt{module () \{ eval \$(lmod bash "\$@")\}}
  \end{itemize}
\end{frame}


\begin{frame}[fragile]
    \frametitle{A Simple Modulefile in Lua}
    {\small
      \begin{alltt}
          help([[A help message for ddt]])
          setenv("TACC_DDT_DIR", "/opt/apps/ddt/3.4.1")
          prepend_path("PATH",   "/opt/apps/ddt/3.4.1/bin")
      \end{alltt}
    }
\end{frame}    

\begin{frame}[fragile]
    \frametitle{Results from \texttt{module load ddt}}
    {\small
      \begin{alltt}
          # in Bash:
          export TACC_DDT_DIR="/opt/apps/ddt/3.4.1"
          export PATH="/opt/apps/ddt/3.4.1/bin:..."

          # in C-shell:
          setenv TACC_DDT_DIR "/opt/apps/ddt/3.4.1"
          setenv PATH         "/opt/apps/ddt/3.4.1/bin:..."
      \end{alltt}
    }
\end{frame}    

\begin{frame}[fragile]
    \frametitle{Results from \texttt{module unload ddt}}
    {\small
      \begin{alltt}
          # in Bash:
          unset TACC_DDT_DIR="/opt/apps/ddt/3.4.1"
          export PATH="..."

          # in C-shell:
          unsetenv TACC_DDT_DIR
          setenv   PATH         "..."
      \end{alltt}
    }
\end{frame}    

\begin{frame}{What is Lmod doing?}
  \begin{itemize}
    \item Sites/Users write actions required for load.
    \item Each function does different things depending on mode.
    \item It is usually either: action, reverse, quiet
  \end{itemize}
\end{frame}

\section{Using Lua Features}

\begin{frame}{How to handle the different modes?}
  \begin{enumerate}
    \item Single functions with if tests
    \item redefine \texttt{setenv}, \texttt{prepend\_path}, ...
    \item Class based solution
  \end{enumerate}
\end{frame}

\begin{frame}[fragile]
    \frametitle{Module functions}
    {\small
      \begin{alltt}
          main()
            local mode = "load"
            mcp = MasterControl.build(mode)
            ...
            sandbox_run(fn)
          end
          function setenv(...)
             mcp:setenv(...)
          end       
          function prepend_path(...)
             mcp:prepend_path(...)
          end       
      \end{alltt}
    }
\end{frame}    

\begin{frame}[fragile]
    \frametitle{Factory to build Lmod evaluation modes}
    {\small
      \begin{alltt}
          function MasterControl.build(name)
            local tbl = \{ load = require('MC_Load'),
                          unload = require('MC_Unload'), \}
            return tbl[name]:create()
          end

      \end{alltt}
    }
\end{frame}

\begin{frame}[fragile]
    \frametitle{MC\_Load.lua, MC\_Unload.lua}
    {\small
      \begin{alltt}
# MC_Load.lua
local M        = inheritsFrom(MasterControl)
M.help         = MasterControl.quiet
M.prepend_path = MasterControl.prepend_path
M.setenv       = MasterControl.setenv
return M

# MC_Unload.lua
local M        = inheritsFrom(MasterControl)
M.help         = MasterControl.quiet
M.prepend_path = MasterControl.remove_path
M.setenv       = MasterControl.unsetenv
return M
      \end{alltt}
    }
\end{frame}    


\begin{frame}{Sandboxing}
  \begin{itemize}
    \item Lmod uses a ``sandbox'' to evaluate modulefiles
    \item This is \texttt{pcall(untrusted\_function)} with a limited
      environment 
    \item No stack traceback with bad code for broken modulefiles.
    \item Protect Users from calling Lmod internal functions.
    \item Sites can add their own functions.
  \end{itemize}
\end{frame}

\begin{frame}{Hooks \& SitePackage.lua}
  \begin{itemize}
    \item Tmod has been around for 20+ years.
    \item Each site does things differently.
    \item Sites must be able to control behavior.
    \item Sites can create a SitePackage.lua file
    \item Register function with the hooks.
  \end{itemize}
\end{frame}

\begin{frame}{Example Hook Functions}
  \begin{itemize}
    \item Use the load hook to keep track of module usage
    \item Register the site's name.
    \item Use the message hook to add to the output of avail and spider.
  \end{itemize}
\end{frame}

\section{Building a Community}

\begin{frame}{Lmod beginning (I)}
  \begin{itemize}
    \item Lmod was first released in 2009.
    \item I prototyped it in Lua
    \item Figuring that Tmod community might be interested.
    \item The prototype was fast enough!
  \end{itemize}
\end{frame}

\begin{frame}{Lmod beginning (II)}
  \begin{itemize}
    \item It was deployed at TACC.
    \item Opt-in/Opt-out strategy.
    \item TACC is one of the largest Open Science HPC systems in US.
    \item With over 10K user accounts (not all active)
    \item Lmod was designed to ``scratch the itch'' of our problems.
    \item Announced Lmod on the Environment Modules Mailing list.
  \end{itemize}
\end{frame}

\begin{frame}{Early User Interaction}
  \begin{itemize}
    \item A user will find Lmod as the ``answer'' to their needs
    \item Sometimes I'm their new best friend.
    \item Sometimes that means stretching Lmod in new directions
    \item Sometimes that means saying no.
    \item I refused to add A.I. or change core features for
      your site only.
  \end{itemize}
\end{frame}

\begin{frame}{Building Trust}
  \begin{itemize}
    \item We are all busy people.
    \item Few will change to my tool just because it is new.
    \item It has to be way ``better'' somehow.
    \item They want to know that you'll be around to support it. 
  \end{itemize}
\end{frame}

\begin{frame}{Lmod's way to Build Trust} 
  \begin{itemize}
    \item A mailing list where questions get answered.
    \item Presentations at important conferences: SC, XSEDE, ISC
    \item Good up-to-date documentation (I'm working on it!!)
    \item Try not to break compatibility with original.
  \end{itemize}
\end{frame}

\begin{frame}{Specific issues w.r.t. Lua}
  \begin{itemize}
    \item Lua is a great language to work in.
    \item But the community of users is much smaller than Python.
    \item There has been some resistance to accepting Lmod
    \item ``\emph{I don't want to learn Lua!}'' - a busy sys-admin
    \item ``\emph{When is this going to be ported to Python?}'' - another sys-admin
    \item A fair amount of feature request and bug reports
    \item Not much contributed code.
  \end{itemize}
\end{frame}

\section{Tech Solutions for reliable programs}

\begin{frame}{Two Tech Solutions}
  \begin{itemize}
    \item A Test Suite for Lmod
    \item New Bug Reports become tests
    \item F
  \end{itemize}
\end{frame}

\end{document}
